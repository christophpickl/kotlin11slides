
\fullimageCapt{kotlin11_banner}{\Large{(language for the \textit{masses})}}{}
% NOTE: 50% JVM, 50% Android (JS now non-experimental/beta anymore)

%\sektion{Introduction}

\frame{\frametitle{Kotlin 1.1 is out} 
  \begin{itemize}[<+->]
    \item Kotlin 1.0 was released 1 year and a bit ago
    \item Release date was 1.3.2017
    \item Current 1.1.1 bugfix release two weeks afterwards
    % 1.0 RC ... 4.2.2016
    % 1.0 ... 15.2.2016
    % 1.0.1 ... 16.3..2016
    % 1.0.2 ... .13.5.2016
    % 1.0.3 ... .30.6.2016
    % 1.0.4 ... 22.9.2016
    % 1.0.5 ... 8.11.2016
    % 1.0.6 ... 27.12.2016
    % 1.0.7 ... 15.3.2017
    % 1.1 RC ... 17.2.2017
    % 1.1 ... 1.3.2017
    % 1.1.1 ... 14.3.2017 bugfix update

    \item 100\% backwards compatible with 1.0    
    % binary compatibility will always be given, but backwards compatibility? like in java for 20 years and carry on all the legacy stuff just to never break user code?
%* 30 people working on kotlin at jetbreains
%* gradle/spring working with kotlin (kobalt FTW)

    \begin{itemize}
    	\item Commitment to \textbf{binary compatibility}
	\item Kotlin \texttt{2.0}? \textit{A possibility, not a plan!} % containing bigger/non-backwards-compatible language changes
	\item Migration tools will be provided % want to integrate a "migration tool", which gets rid of warnings when using old language constructs
    \end{itemize}
       
    \item Currently working on \href{https://blog.jetbrains.com/kotlin/2017/04/kotlinnative-tech-preview-kotlin-without-a-vm/}{kotlin-native}
    % (via LLVM) (seems kind like a problem to them to support JVM/JS/native, keep it to a minimum and add extensions)
    % 

  \end{itemize}
}

\frame{\frametitle{JVM News}
  \begin{itemize}[<+->]
    \item java 8
    \item std libs
    \item compiler option
  \end{itemize}
}

\frame{\frametitle{Overview}
  \begin{itemize}[<+->]
    \item Coroutines
    \item ...
    \item JavaScript
  \end{itemize}
}
