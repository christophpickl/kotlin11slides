
%\fullimageCapt{lighthouse}{A lighthouse on \href{https://en.wikipedia.org/wiki/Kotlin_Island}{Kotlin Island}, Russia}{}

\sektion{Language Features}

%%%%%%%%%%%%%%%%%%%%%%%%%%%%%%%%%%%%%%%%%%%%%%%%%%%%%%%%%%%%
\begin{frame}\frametitle{What we'll cover}
  \begin{enumerate}[<+->]
  	\item Async processing with \textbf{coroutines}
	\item Define own \textbf{type aliases}
	\item Bound callable \textbf{references}
	\item Improvements for \textbf{data} and \textbf{sealed} classes
	\item \textbf{Destructuring} in lambdas
	\item Local \textbf{delegated} properties
	\item \textbf{Underscore} for numbers and parameters % (lambdas, destructuring declarations)
	\item Type inference and inlining for \textbf{properties} % type inference for getter properties = shorter syntax :)
	\item Generic \textbf{enum} value access
	\item Restrict lambda scope with \texttt{@DslMarker}
  \end{enumerate}
\end{frame}


%%%%%%%%%%%%%%%%%%%%%%%%%%%%%%%%%%%%%%%%%%%%%%%%%%%%%%%%%%%%

\begin{frame}[fragile] \frametitle{Coroutines are \ldots}
\begin{itemize}[<+->]
	\item an \textbf{expressive} tool for implementing \textbf{asynchronous} behavior
	\begin{itemize}
		\item look like regular, sequential function invocations
	\end{itemize}
	\item very \textbf{lightweight} threads (like \href{https://en.wikipedia.org/wiki/Fiber_(computer_science)}{fibers}) % composable/delimited [stackless] ?ontinuations
	\item still marked as \textbf{experimental}
	\item very \textbf{low-level} designed so that frameworks can build upon it
	\begin{itemize}
		\item extensions for Android, JavaFX, Swing, \ldots
		\item like \texttt{async/await} or \texttt{generators/yield}
		\item Reference implementation: \texttt{kotlinx-coroutines-*}
		% kotlinx-coroutines-core, kotlinx-coroutines-jdk8, kotlinx-coroutines-nio
		% kotlinx-coroutines-reactive, kotlinx-coroutines-rx1, kotlinx-coroutines-rx2
		% kotlinx-coroutines-android, kotlinx-coroutines-javafx, kotlinx-coroutines-swing
	\end{itemize}
\end{itemize}
\end{frame}

\begin{frame}[fragile] \frametitle{Computation with lots of coroutines}
Coroutines scale pretty good processing loads of async operations.\pause
\begin{lstlisting}
val jobs = List(100_000)|\pause| {
  async(CommonPool)|\pause| {
    delay(1000L)
    1
  }
}|\pause|

runBlocking { // bridge async world|\pause|
  println(jobs.sumBy {
    it.await()
  })
}
\end{lstlisting}
\end{frame}

\begin{frame}[fragile] \frametitle{Computation with lots of threads}
The same code with good old threads will \ldots
\begin{lstlisting}
for (i in 1..100_000) {
  thread(start = true) {
    Thread.sleep(1000)
  }
}|\pause|
\end{lstlisting}
\texttt{\textcolor{red}{OutOfMemoryError}}: unable to create new native thread!
\end{frame}

\begin{frame}[fragile] \frametitle{Async processing without coroutines}
\begin{lstlisting}
fun load(callback: (Int) -> Unit) {
    sleep(1000L)
    callback(21)
}|\pause|
fun process(nr: Int, cb: (Int) -> Unit) {
    sleep(1000L)
    cb(nr * 2)
}|\pause|
fun main() {
    load { loaded ->
        process(loaded) { processed ->
            // processed == 42
        }
    }
}
\end{lstlisting}
\end{frame}

\fullimageCapt{callback_hell}{When the so-called ``callback hell'' strikes you hard}{10cm}
% Source: \url{https://collinmakersquare.wordpress.com}

\fullimageCapt{functions_inside}{}{9cm}

\fullimageCapt{coroutines_processing}{Suspendable computation explained}{10cm}

\begin{frame}[fragile] \frametitle{Async processing with coroutines}
\begin{lstlisting}
suspend fun load(): Int {
    delay(1000L)
    return 21
}|\pause|
suspend fun process(nr: Int): Int {
    delay(1000L)
    return nr * 2
}|\pause|

fun main() = runBlocking {
    val nr = process(load())
    // blocking or use async() and await()
}
\end{lstlisting}
\end{frame}

\begin{frame}[fragile] \frametitle{Async sequences}
\begin{lstlisting}
val nrs = buildSequence {
  yieldAll(2.rangeTo(4))
  sleep(1000L)
  yield(42)
}|\pause|
nrs.forEach(::println)
// 2, 3, 4, ... 42|\pause|

// fun <T> buildIterator(
//   action: suspend SequenceBuilder<T>.() 
//     -> Unit): Iterator<T>
\end{lstlisting}
\end{frame}

\fullimageCapt{two_programmers}{}{9cm}

%%%%%%%%%%%%%%%%%%%%%%%%%%%%%%%%%%%%%%%%%%%%%%%%%%%%%%%%%%%%

\begin{frame}[fragile] \frametitle{Type aliases}
Add improved readability through custom naming.\pause
\begin{lstlisting}
fun request(expectedStatusCode: Int) {}|\pause|
typealias StatusCode = Int|\pause|
fun request(expected: StatusCode) {}
request(200)|\pause|

fun subscribe(listener : (Event) -> Unit) {}|\pause|
typealias Listener = (Event) -> Unit
fun subscribe(listener : Listener) {}|\pause|

class MustBeTopLevel {
  typealias Nope = Double // compile error
}
\end{lstlisting}
\smallnote{\keep{https://github.com/Kotlin/KEEP/issues/4}}
\end{frame}

%%%%%%%%%%%%%%%%%%%%%%%%%%%%%%%%%%%%%%%%%%%%%%%%%%%%%%%%%%%%

\begin{frame}[fragile] \frametitle{Bound callable references}
Reference a member of an object instance.
\begin{lstlisting}
val numberRegex = "\\d+".toRegex()
val list = listOf("a", "1")|\pause|

// before kotlin11
list.filter { numberRegex.matches(it) }
  .forEach(::println)|\pause|

// with kotlin11
list.filter(numberRegex::matches)
  .forEach(::println)
\end{lstlisting}
\smallnote{\keep{https://github.com/Kotlin/KEEP/issues/5}}
\end{frame}

%%%%%%%%%%%%%%%%%%%%%%%%%%%%%%%%%%%%%%%%%%%%%%%%%%%%%%%%%%%%

\begin{frame}[fragile] \frametitle{Improved data and sealed classes}
Subtypes of a \texttt{sealed} class can now reside outside the class and \texttt{data} class can extend other classes.
\begin{lstlisting}
sealed class Expression|\pause|

// sealed subs can be declared outside
object Operator : Expression()|\pause|

// a data class can now extend another class
data class Operand(val symbol: String)
  : Expression()
  
// use an Expression with a nice when()
\end{lstlisting}
\end{frame}

%%%%%%%%%%%%%%%%%%%%%%%%%%%%%%%%%%%%%%%%%%%%%%%%%%%%%%%%%%%%

\begin{frame}[fragile] \frametitle{Destructuring in lambdas}
Destructuring now works in lambdas (and for data class).
\begin{lstlisting}
val map = mapOf(1 to "one")|\pause|

// before kotlin11
map.mapValues { entry ->
  val (key, value) = entry
  "$key = $value"
}|\pause|

// with kotlin11
map.mapValues { (key, value) -> 
  "$key = $value" }
\end{lstlisting}
\smallnote{\keep{https://github.com/Kotlin/KEEP/blob/master/proposals/destructuring-in-parameters.md}}
\end{frame}

%%%%%%%%%%%%%%%%%%%%%%%%%%%%%%%%%%%%%%%%%%%%%%%%%%%%%%%%%%%%

\begin{frame}[fragile] \frametitle{Local delegated properties}
Not only for (class) properties, anymorebut also for local variables.
\begin{lstlisting}
fun exec(stringProvider: () -> String) {
  val string by lazy(stringProvider)|\pause|
  
  // short circuit evaluation FTW
  if (condition() && string.isValid()) {
    // computed result will be cached
    println(string)
  }
}
\end{lstlisting}
\smallnote{\keep{https://github.com/Kotlin/KEEP/blob/master/proposals/local-delegated-properties.md}}
\end{frame}

%%%%%%%%%%%%%%%%%%%%%%%%%%%%%%%%%%%%%%%%%%%%%%%%%%%%%%%%%%%%

\begin{frame}[fragile] \frametitle{Underscores}
Underscore for numeral literals and unused parameters.  % (lambdas, destructuring declarations)
\begin{lstlisting}
// more readable numbers
val creditCardNumber = 1234_5678_9012_3456L
val hexBytes = 0xFF_EC_DE_5E
val hexWords = 0xCAFE_BABE
val maxval = 0x7fff_ffff_ffff_ffffL
val bytes = 0b0110_1001;|\pause|

// explicitly declare unused params
// generating lots of warnings for old code
mapOf(1 to "one").map { (k, _) -> k }
\end{lstlisting}
\smallnote{\keep{https://github.com/Kotlin/KEEP/blob/master/proposals/underscores-in-numeric-literals.md}}
\end{frame}

%%%%%%%%%%%%%%%%%%%%%%%%%%%%%%%%%%%%%%%%%%%%%%%%%%%%%%%%%%%%

\begin{frame}[fragile] \frametitle{Enhanced Properties}
Type inference and inlining for \textbf{properties} % type inference for getter properties = shorter syntax :)
\begin{lstlisting}
data class Person(val age: Int) {|\pause|
  // before kotlin11
  val isAdult get(): Boolean = age >= 18|\pause|

  // with kotlin11
  val isAdult get() = age >= 18
}|\pause|

val <T> List<T>.lastIndex: Int
    inline get() = this.size - 1
\end{lstlisting}
\smallnote{\keep{https://github.com/Kotlin/KEEP/blob/master/proposals/inline-properties.md}}
\end{frame}

%%%%%%%%%%%%%%%%%%%%%%%%%%%%%%%%%%%%%%%%%%%%%%%%%%%%%%%%%%%%

\begin{frame}[fragile] \frametitle{Generic enum value access}
Access all/any enum and specify type via generics.
\begin{lstlisting}
enum class RGB { RED, GREEN, BLUE }|\pause|

val red: RGB = enumValueOf("RED")|\pause|
val rgbs: Array<RGB> = enumValues()|\pause|

println(enumValues<RGB>()
  .joinToString(transform = RGB::name))
  // RED, GREEN, BLUE
\end{lstlisting}
\smallnote{\keep{https://github.com/Kotlin/KEEP/blob/master/proposals/generic-values-and-valueof-for-enums.md}}
\end{frame}

%%%%%%%%%%%%%%%%%%%%%%%%%%%%%%%%%%%%%%%%%%%%%%%%%%%%%%%%%%%%

\begin{frame}[fragile] \frametitle{DSLs -- The Problem}
The inner scope always inherits the context of the outer scope.\pause
\begin{lstlisting}
html {
  head {
    head {
      // this should not be able
      // as it makes no sense
    }
  }
}
\end{lstlisting}\pause
When within \texttt{head} let's remove the context of \texttt{html}!
\end{frame}


\begin{frame}[fragile] \frametitle{Casual implementation of HTML DSL}
\begin{lstlisting}
fun html(code: HtmlContext.() -> Unit) {
  code(HtmlContext())
}|\pause|
class HtmlContext {
  fun head(code: HeadContext.() -> Unit) {
    code(HeadContext())
  }
}|\pause|
class HeadContext { }
\end{lstlisting}
\end{frame}


\begin{frame}[fragile] \frametitle{Restrict scope with @DslMarker}
\begin{lstlisting}
fun html(code: HtmlContext.() -> Unit) {
  code(HtmlContext())
}|\pause|
@DslMarker annotation class MyDslMarker|\pause|
@MyDslMarker class HtmlContext {
  fun head(code: HeadContext.() -> Unit) {
    code(HeadContext())
  }
}|\pause|
@MyDslMarker class HeadContext { }
\end{lstlisting}
\smallnote{\keep{https://github.com/Kotlin/KEEP/blob/master/proposals/scope-control-for-implicit-receivers.md}}
\end{frame}

\begin{frame}[fragile] \frametitle{Humans still remain in control}
If you still want to access the \texttt{html} context you can still do so.
\begin{lstlisting}
html {
  head {
    // head { } compile error!
    this@html.head { } // enforce
  }
}
\end{lstlisting}
\end{frame}

