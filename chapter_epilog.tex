
%\fullimageCapt{lighthouse}{A lighthouse on \href{https://en.wikipedia.org/wiki/Kotlin_Island}{Kotlin Island}, Russia}{}

\fulltitle{That's it}

\frame{\frametitle{Further Reading} 
  \begin{itemize}[<+->]
    \item Slides and sources on GitHub \\ \href{https://github.com/christophpickl/kotlin11slides}{https://github.com/.../kotlin11slides}
    %https://github.com/christophpickl/kotlin11slides/raw/master/bin/Kotlin11-2017_04_18-At_KUG.pdf
    \item Kotlin 1.1 sample code \\ \href{https://github.com/christophpickl/awesomekotlin/blob/master/src/main/kotlin/com/github/christophpickl/awesomekotlin/kotlin11/}{https://github.com/.../awesomekotlin/kotlin11}
    \item Official Release Page \\ \small{\url{https://kotlinlang.org/docs/reference/whatsnew11.html}}
    \item Andrey demos Kotlin 1.1 \\ \url{https://youtube.com/watch?v=zpyJHSR-5ts}
  \end{itemize}
  
  \vspace{1cm}
  \textit{Have a nice Kotlin!}
}


\fulltitle{One more thing \ldots}

\begin{frame}[fragile] \frametitle{Logging, the Kotlin way}

First declare a Gradle dependency (Slf4j extensions):
\begin{lstlisting}
compile
 "io.github.microutils:kotlin-logging:1.4.4"
\end{lstlisting}
\pause

Write your own shortcut function:
\begin{lstlisting}
fun LOG(func: () -> Unit) =
  KotlinLogging.logger(func)
// define a code template in your IDE
\end{lstlisting}
\pause

Simple usage:
\begin{lstlisting}
class Foo {
  private val log = LOG {}
  init {
    log.debug { "lazy evaluated $this" }
  }
}
\end{lstlisting}
\pause

\end{frame}
