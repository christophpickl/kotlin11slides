
\fulltitle{That's it}

%%%%%%%%%%%%%%%%%%%%%%%%%%%%%%%%%%%%%%%%%%%%%%%%%%%%%%%%%%%%

\frame{\frametitle{Further Reading} 
  \begin{itemize}[<+->]
    \item Slides and sources \\ \small{\texttt{\href{https://github.com/christophpickl/kotlin11slides}{https://github.com/.../kotlin11slides}}}
    %https://github.com/christophpickl/kotlin11slides/raw/master/bin/Kotlin11-2017_04_18-At_KUG.pdf
    \item Some sample code \\ \small{\texttt{\href{https://github.com/christophpickl/awesomekotlin/blob/master/src/main/kotlin/com/github/christophpickl/awesomekotlin/kotlin11/}{https://github.com/.../awesomekotlin/kotlin11}}}
    \item Official release page \\ \small{\url{https://kotlinlang.org/docs/reference/whatsnew11.html}}
    \item \textbf{K}otlin \textbf{E}volution and \textbf{E}nhancement \textbf{P}rocess \\ \small{\url{https://github.com/Kotlin/KEEP}}
    \item Kotlin Vienna Usergroup \\ \small{\url{https://www.meetup.com/Kotlin-Vienna/}}
  \end{itemize}
}

\fullimageCapt{youtube_kotlin11}{\url{https://youtube.com/watch?v=zpyJHSR-5ts}}{10cm}

\fullimageCapt{kotlin_in_action}{Get your very own copy, now!}{5cm}

%%%%%%%%%%%%%%%%%%%%%%%%%%%%%%%%%%%%%%%%%%%%%%%%%%%%%%%%%%%%

\fulltitle{One more thing \ldots}


\begin{frame}[fragile] \frametitle{Logging, the Kotlin way}
First declare a Gradle dependency (kind-a Slf4j extension):
\begin{lstlisting}
compile
 "io.github.microutils:kotlin-logging:1.4.4"
\end{lstlisting}
\pause

Write your own shortcut function:
\begin{lstlisting}
fun LOG(func: () -> Unit) =
  KotlinLogging.logger(func)
// define a code template in your IDE
\end{lstlisting}
\pause

Simple usage:
\begin{lstlisting}
class Foo {
  private val log = LOG {}
  init {
    log.debug { "lazy evaluated $this" }
  }
}
\end{lstlisting}
\pause

\end{frame}

%%%%%%%%%%%%%%%%%%%%%%%%%%%%%%%%%%%%%%%%%%%%%%%%%%%%%%%%%%%%

\fulltitle{\textit{Have a nice Kotlin} \texttt{:\}}}
