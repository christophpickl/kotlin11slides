
\sektion{Standard Library}
% NOTE: 50% JVM, 50% Android (JS now non-experimental/beta anymore)
	
\frame{\frametitle{What we'll cover}
  \begin{enumerate}[<+->]
    \item Nullable number conversion
    \item \texttt{also()}, \texttt{takeIf()}, \texttt{takeUnless()}
    \item \texttt{minOf()}, \texttt{maxOf()}
    \item \texttt{onEach()}
    \item \texttt{groupingBy()}
    \item Map functions % Map.toMap(), Map.toMutableMap(), Map.minus(), Map.getValue(), Map.withDefault()
    \item List comprehension
    \item Array manipulation
    \item Base classes for collections
    \item \texttt{mod} renamed to \texttt{rem}
  \end{enumerate}
}

%%%%%%%%%%%%%%%%%%%%%%%%%%%%%%%%%%%%%%%%%%%%%%%%%%%%%%%%%%%%
\begin{frame}[fragile] \frametitle{Nullable number conversion}
Conversion avilable for: Byte, Short, Int, Long, Float, Double
\begin{lstlisting}
"x".toIntOrNull() ?: 42 // = 42

// delegates to java.lang.Integer.parseInt
"x".toInt() // NumberFormatException|\pause|

val radix = 2 // 2..36
// radix2 = 101010, radix16 = 2a
println(42.toString(radix))
// radix2 = 3, radix16 = 17
println("11".toIntOrNull(radix))
\end{lstlisting}
\smallnote{\keep{https://github.com/Kotlin/KEEP/blob/master/proposals/stdlib/string-to-number.md}}
\end{frame}


%%%%%%%%%%%%%%%%%%%%%%%%%%%%%%%%%%%%%%%%%%%%%%%%%%%%%%%%%%%%
\begin{frame}[fragile] \frametitle{\texttt{also()}}
Same as \texttt{apply()} but without changing the \texttt{this} reference.
% could store this reference of a1 as a workaround
\begin{lstlisting}
val a1 = "a".apply {
  "b".apply {
    this // "b"
    @Suppress("LABEL_NAME_CLASH")
    this@apply // "b" but want "a" :(
  }
}|\pause|
val a2 = "a".also { outer ->
  "b".also {
    it // "b"
    outer // "a" :)
  }
}
\end{lstlisting}
\end{frame}

%%%%%%%%%%%%%%%%%%%%%%%%%%%%%%%%%%%%%%%%%%%%%%%%%%%%%%%%%%%%

\begin{frame}[fragile] \frametitle{\texttt{takeIf()} and \texttt{takeUnless()}}
\texttt{takeIf()} is like \texttt{filter()} but acts on a singlue value and returns \texttt{null} on mismatch. \texttt{takeUnless()} simply inverts the condition.
\begin{lstlisting}
val file = File("path")
if (!file.exists()) {
  return false
}|\pause|

// takeIf works well with elvis operator
val file = File("path").takeIf(File::exists)
  ?: return false
\end{lstlisting}
\end{frame}

%%%%%%%%%%%%%%%%%%%%%%%%%%%%%%%%%%%%%%%%%%%%%%%%%%%%%%%%%%%%
\begin{frame}[fragile] \frametitle{Keyword highlighter using \texttt{takeIf()}}
\begin{lstlisting}
val input = "Kotlin"
val keyword = "in"|\pause|

val index = input.indexOf(keyword)
    .takeIf { it >= 0 } ?: error()|\pause|
//  .takeUnless { it < 0 } ?: error()|\pause|

println("'$keyword' was found in '$input'")
println(input)
println(" ".repeat(index) + "^")
// 'in' was found in 'Kotlin'
// Kotlin
//     ^
\end{lstlisting}
\end{frame}


%%%%%%%%%%%%%%%%%%%%%%%%%%%%%%%%%%%%%%%%%%%%%%%%%%%%%%%%%%%%
\begin{frame}[fragile] \frametitle{\texttt{minOf()} and \texttt{maxOf()}}
\begin{lstlisting}
val l1 = listOf("one")
val l2 = listOf("0", "1")
val l3 = listOf("x", "y", "z")

// delegates to Math.min for 2 values
minOf(l1.size, l2.size)
Math.min(l1.size, l2.size)

minOf(l1.size, l2.size, l3.size)
Math.min(l1.size, Math.min(l2.size, l3.size))

// minOf/maxOf supports only 2-3 values :-/

minOf(l1, l2, compareBy { it.size })
\end{lstlisting}
So why not simply use: \texttt{Iterable<T>.min(): T?}
\end{frame}

%%%%%%%%%%%%%%%%%%%%%%%%%%%%%%%%%%%%%%%%%%%%%%%%%%%%%%%%%%%%
\begin{frame}[fragile] \frametitle{\texttt{onEach()}}
\footnotesize{\texttt{fun <T, I : Iterable<T>> I.onEach(action: (T) -> Unit): I}} \\
\footnotesize{\texttt{fun <T> Iterable<T>.forEach(action: (T) -> Unit): Unit}}
\normalsize{}
\begin{lstlisting}
listOf("foobar", "foo")
  .filter { it.endsWith("bar") }
  // chain item processing
  .onEach { println("Found item: $it") }
  .forEach { /* finally operate on them */ }
\end{lstlisting}
\smallnote{\keep{https://github.com/Kotlin/KEEP/blob/master/proposals/stdlib/on-each.md}, same as \texttt{apply \{ forEach \{ \} \}}}
\end{frame}

%%%%%%%%%%%%%%%%%%%%%%%%%%%%%%%%%%%%%%%%%%%%%%%%%%%%%%%%%%%%
\begin{frame}[fragile] \frametitle{\texttt{groupingBy()}}
\footnotesize{\texttt{fun <T, K> Iterable<T>.groupingBy(key: (T) -> K): Grouping<T, K>}} \\
\footnotesize{\texttt{fun <T, K> Iterable<T>.groupBy(key: (T) -> K): Map<K, List<T>>}}
\begin{lstlisting}
val list = listOf("anna", "otto", "oscar")

list
  .groupBy(String::first)
  .mapValues { (_, list) -> list.size } // a=1, o=2
  // creates intermediate map|\pause|

list
  .groupingBy(String::first)
  .eachCount() // invokes foldTo()
\end{lstlisting}
\end{frame}

%%%%%%%%%%%%%%%%%%%%%%%%%%%%%%%%%%%%%%%%%%%%%%%%%%%%%%%%%%%%
\begin{frame}[fragile] \frametitle{Map functions}
\texttt{toMap()}, \texttt{toMutableMap()}, minus, \texttt{getValue()}, \texttt{withDefault()}
\begin{lstlisting}
var map = mapOf("x" to 1)
map.toMap() // create a copy
map.toMutableMap() // create a mutable copy

// plus already worked
map += ("y" to 2)
// now minus also supported
map -= "y"

map.getValue("y") // throws
val map2 = map.withDefault { "!\$it!" }
map2.getValue("y") // !y!
\end{lstlisting}
\smallnote{\keep{https://github.com/Kotlin/KEEP/blob/master/proposals/stdlib/map-copying.md}}
 % AD MINUS: loses ability to access previous value
% AD getValue: NoSuchElementException.message: "Key two is missing in the map."
%                       would be nice, if there was a custom exception type which got the key as an accessible field (string parsing, really?!)
\end{frame}

%%%%%%%%%%%%%%%%%%%%%%%%%%%%%%%%%%%%%%%%%%%%%%%%%%%%%%%%%%%%
\begin{frame}[fragile] \frametitle{Something I'm missing here \ldots}
Create a mutable (!) map based on a list of pairs.
\begin{lstlisting}
val list: Pair<String, Int> = listOf("x" to 1)
list.toMap() // already existing
list.toMutableMap() // does NOT exist|\pause|

// let's workaround
fun <K, V> Iterable<Pair<K, V>>.toMutableMap(): MutableMap<K, V> {
  val immutableMap = toMap()
  val map = HashMap<K, V>(immutableMap.size)
  map.putAll(immutableMap)
  return map
}
\end{lstlisting}
\end{frame}

%%%%%%%%%%%%%%%%%%%%%%%%%%%%%%%%%%%%%%%%%%%%%%%%%%%%%%%%%%%%
\begin{frame}[fragile] \frametitle{List comprehension}
\footnotesize{\texttt{fun <T> List(size: Int, init: (index: Int) -> T): List<T>}}
\begin{lstlisting}
// already existed for arrays
IntArray(4) { it * 2 }.toList()
// [0, 2, 4, 6]|\pause|

// now for lists as well
List(4) { it * 2 }
MutableList(4) { it * 2 }
\end{lstlisting}
\small{(Still not the same as in Haskell)}
\end{frame}

%%%%%%%%%%%%%%%%%%%%%%%%%%%%%%%%%%%%%%%%%%%%%%%%%%%%%%%%%%%%
\begin{frame}[fragile] \frametitle{Array manipulation}
New methods: \texttt{content[Deep](Equals|HashCode|ToString)}
\begin{lstlisting}
val a1 = arrayOf("a", "b")
val a2 = arrayOf(arrayOf("a"), arrayOf("b"))

a1.toString() // [Ljava.lang.String;\@1b3af
a1.contentToString() // [a, b]
a2.contentDeepToString() // [[a], [b]]

// ... equals, hashCode the same ...
\end{lstlisting}
\small{(Actually just a delegation to \texttt{java.util.Arrays})}
\end{frame}

%%%%%%%%%%%%%%%%%%%%%%%%%%%%%%%%%%%%%%%%%%%%%%%%%%%%%%%%%%%%
\begin{frame}[fragile] \frametitle{Base classes for collections}
New classes: \texttt{Abstract[Mutable](Collection|List|Set|Map)} \\
% base classes from JDK are not truly immutable; throw runtime exception; and sometimes deal with platform type String!
\begin{lstlisting}
// skeletal implementation of [List]
val listWithOneElement: List<String> =
  object : AbstractList<String>() {
    override val size: Int
      get() = 1
    override fun get(index: Int): String {
      return "always foo"
    }
  }
\end{lstlisting}
\smallnote{See: \keep{https://github.com/Kotlin/KEEP/blob/master/proposals/stdlib/abstract-collections.md}, \href{https://github.com/JetBrains/kotlin/tree/master/libraries/stdlib/src/kotlin/collections}{stdlib sources}}
\end{frame}

%%%%%%%%%%%%%%%%%%%%%%%%%%%%%%%%%%%%%%%%%%%%%%%%%%%%%%%%%%%%
% https://youtrack.jetbrains.com/issue/KT-14650 ... 
\begin{frame}[fragile] \frametitle{\texttt{mod} renamed to \texttt{rem}}
 \texttt{mod} function on integral types is inconsistent with \texttt{BigInteger}:
\begin{lstlisting}
val minus3 = BigInteger.valueOf(-3)
val plus5 = BigInteger.valueOf(5)

minus3.mod(plus5) // 2
% Int.mod() was deprecated
minus3.toInt().mod(plus5.toInt()) // -3

minus3.rem(plus5) // -3
minus3.toInt().rem(plus5.toInt()) // -3
\end{lstlisting}
\small{(Math nerds know their \href{https://www.microsoft.com/en-us/research/wp-content/uploads/2016/02/divmodnote-letter.pdf}{ecuclidean rings})}
\end{frame}
